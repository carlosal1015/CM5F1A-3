\section{Pregunta N$^{\circ}$6\qquad León Alonzo Terrones Caccha}

\begin{frame}
    \begin{enumerate}\setcounter{enumi}{5}
        \item

               Encuentre las constantes $c_0$, $c_1$ y $x_1$ de modo que la fórmula de cuadratura
              \begin{math}
                  \displaystyle
                  \int\limits_{0}^{1}
                  f(x)=c_0f(0)+c_1f(x_1)
              \end{math}
              tenga el grado de precisión más alto posible.

            
    \end{enumerate}

    \begin{solution}
        Sea $f$ constante de valor $K$. Entonces
        \begin{align*}
        \displaystyle\int\limits_{0}^{1}f(x)\,dx&=K=c_0K+c_1K
        \end{align*}
   Por tanto 
   \begin{equation}\label{1}
       c_0K+c_1=1
   \end{equation}
        Para $f(x)=ax+b$, i.e. una función lineal:
        \begin{align*}
        \displaystyle\int\limits_{0}^{1}f(x)\,dx&=\displaystyle\int\limits_{0}^{1}ax+b\,dx=\frac{a}{2}+b=c_0f(0)+c_1f(x_1)=c_0b+c_1(ax_1+b)\underbrace{=}_{\ref{1}}ac_1x_1+b
        \end{align*}
        Por tanto se tiene
        \begin{equation*}
            c_1=\frac{1}{2x_1}
        \end{equation*}
        y 
        \begin{equation*}
            c_0=1-\frac{1}{2x_1}
        \end{equation*}
        para cualquier $x_1\in [0,1]$.

        
    \end{solution}
\end{frame}




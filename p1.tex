1. Suponga que $N(h)$ es una aproximación para $M$ para cada $h>0$ y que
$$
M=N(h)+k_1 h+K_2 h^2+K_3 h^3+\cdots
$$
para algunas constantes $K_1, K_2, K_3, \cdots$. Utilice los valores $N(h), N\left(\frac{h}{3}\right)$ y $N\left(\frac{h}{9}\right)$ para producir una aproximación $\mathcal{O}\left(h^3\right)$ de $M$.
\section{Pregunta N$^{\circ}$1\qquad Carlos Alonso Aznarán Laos}

\begin{frame}
    \begin{enumerate}\setcounter{enumi}{0}
        \item

              Suponga que $N\left(h\right)$ es una \emph{aproximación}
              de $M$ y
              \begin{equation*}
                  \forall h>0:
                  M=
                  N\left(h\right)+
                  K_{1}h+
                  K_{2}h^{2}+
                  K_{3}h^{3}+
                  \cdots
              \end{equation*}
              existen
              \begin{math}
                  \left\{
                  K_{1},
                  K_{2},
                  K_{3},
                  \dotsc
                  \right\}\subset
                  \mathbb{R}
              \end{math}.
              Encuentre una aproximación de $M$ con orden
              $\mathcal{O}\left(h^3\right)$, en función de
              \begin{math}
                  N\left(h\right),
              \end{math},
              \begin{math}
                  N\left(\dfrac{h}{3}\right)
              \end{math}
              y
              \begin{math}
                  N\left(\dfrac{h}{9}\right)
              \end{math}.
    \end{enumerate}

    \begin{solution}
        .
    \end{solution}
\end{frame}

\begin{frame}
    \frametitle{Extrapolación de Richardson}

    .
\end{frame}

\begin{frame}
    \begin{solution}
        .
    \end{solution}
\end{frame}
\section{Pregunta N$^{\circ}$1\qquad Carlos Alonso Aznarán Laos}

\begin{frame}
    \frametitle{Extrapolación de Richardson}

    Suponga que $\forall h\neq0$ tenemos una fórmula $N_{1}\left(h\right)$
    que aproxima un valor desconocido $M$
    \begin{equation*}
        M-
        N_{1}\left(h\right)=
        K_{1}h+
        K_{2}h^{2}+
        K_{3}h^{3}+
        \cdots
    \end{equation*}
    existen
    \begin{math}
        \left\{
        K_{1},
        K_{2},
        K_{3},
        \dotsc
        \right\}\subset
        \mathbb{R}
    \end{math}.
    Si $K_{1}\neq0$, entonces el \emph{error de truncamiento} es
    $\mathcal{O}\left(h\right)$.

    \

    \begin{example}
        \begin{align*}
            f^{\prime}\left(x\right)-
            \dfrac{f\left(x+h\right)-f\left(x\right)}{h}    & =
            -\dfrac{f^{\left(2\right)}\left(x\right)}{2!}h
            -\dfrac{f^{\left(3\right)}\left(x\right)}{3!}h^{2}
            -\dfrac{f^{\left(4\right)}\left(x\right)}{4!}h^{3}
            -\cdots.                                            \\
            f^{\prime}\left(x\right)-
            \dfrac{f\left(x+h\right)-f\left(x-h\right)}{2h} & =
            -\dfrac{f^{\left(3\right)}\left(x\right)}{3!}h^{2}
            -\dfrac{f^{\left(5\right)}\left(x\right)}{5!}h^{4}
            -\dfrac{f^{\left(7\right)}\left(x\right)}{7!}h^{6}
            -\cdots.
            \shortintertext{En general,}
            f^{\prime}\left(x\right)-
            N_{j}\left(h\right)                             & =
            \mathcal{O}\left(h^{2j}\right),
            \shortintertext{donde}
            N_{j}\left(h\right)                             & =
            N_{j-1}\left(\dfrac{h}{2}\right)+
            \dfrac{
            N_{j-1}\left(\frac{h}{2}\right)-N_{j-1}\left(h\right)
            }{
            4^{j-1}-1
            }.
        \end{align*}
    \end{example}
\end{frame}

\begin{frame}
    \begin{enumerate}\setcounter{enumi}{0}
        \item

              Suponga que $N\left(h\right)$ es una \emph{aproximación}
              de $M$ y que para cualquier tamaño de paso $h>0$
              \begin{equation}\label{eq:richardson}
                  M=
                  N\left(h\right)+
                  K_{1}h+
                  K_{2}h^{2}+
                  K_{3}h^{3}+
                  \cdots
              \end{equation}
              existen
              \begin{math}
                  \left\{
                  K_{1},
                  K_{2},
                  K_{3},
                  \dotsc
                  \right\}\subset
                  \mathbb{R}
              \end{math}.
              Encuentre una aproximación
              $\mathcal{O}\left(h^3\right)$ de $M$ en términos de
              \begin{math}
                  N\left(h\right)
              \end{math},
              \begin{math}
                  N\left(\dfrac{h}{3}\right)
              \end{math}
              y
              \begin{math}
                  N\left(\dfrac{h}{9}\right)
              \end{math}.
    \end{enumerate}

    \begin{solution}
        Evaluamos~\eqref{eq:richardson} para $\alert{h}$,
        $\alert{\dfrac{h}{3}}$ y $\alert{\dfrac{h}{9}}$:
        \begin{align*}
            M & =
            N\left(h\right)+
            K_{1}\alert{h}+
            K_{2}\alert{h^{2}}+
            K_{3}\alert{h^{3}}+
            \cdots
            \\
            M & =
            N\left(\alert{\dfrac{h}{3}}\right)+
            K_{1}\alert{\dfrac{h}{3}}+
            K_{2}\alert{\dfrac{h^{2}}{9}}+
            K_{3}\alert{\dfrac{h^{3}}{27}}+
            \cdots \\
            M & =
            N\left(\alert{\dfrac{h}{9}}\right)+
            K_{1}\alert{\dfrac{h}{9}}+
            K_{2}\alert{\dfrac{h^{2}}{81}}+
            K_{3}\alert{\dfrac{h^{3}}{729}}+
            \cdots
        \end{align*}
        Multiplicamos la primera ecuación por $\alert{A}$, la segunda
        por $\alert{B}$, y la tercera por $\alert{C}$.
        \begin{align*}
            \alert{A}M & =
            \alert{A}N\left(h\right)+
            \alert{A}K_{1}h+
            \alert{A}K_{2}h^{2}+
            \alert{A}K_{3}h^{3}+
            \cdots
            \\
            \alert{B}M & =
            \alert{B}N\left(\dfrac{h}{3}\right)+
            \alert{B}K_{1}\dfrac{h}{3}+
            \alert{B}K_{2}\dfrac{h^{2}}{9}+
            \alert{B}K_{3}\dfrac{h^{3}}{27}+
            \cdots         \\
            \alert{C}M & =
            \alert{C}N\left(\dfrac{h}{9}\right)+
            \alert{C}K_{1}\dfrac{h}{9}+
            \alert{C}K_{2}\dfrac{h^{2}}{81}+
            \alert{C}K_{3}\dfrac{h^{3}}{729}+
            \cdots
        \end{align*}
    \end{solution}
\end{frame}

\begin{frame}
    \begin{solution}
        Sumando y cancelando $K_{1}$ y $K_{2}$ resulta las ecuaciones
        \begin{align*}
            A+\dfrac{B}{3}+\dfrac{C}{9}  & =0. \\
            A+\dfrac{B}{9}+\dfrac{C}{81} & =0.
            \shortintertext{Restando da}
            \dfrac{2B}{9}+
            \dfrac{8C}{81}               & =
            0.
            \shortintertext{Multiplicando por $81$}
            18B+
            8C                           & =
            0.                                 \\
            9B+
            4C                           & =
            0.
            \shortintertext{Haciendo $C=0$, $B=1$ y $A=-9$. Por lo tanto,}
            \left(\alert{A}+\alert{B}+\alert{C}\right)
            M                            & =
            \alert{A}N\left(h\right)+
            \alert{B}N\left(\dfrac{h}{3}\right)+
            \alert{C}N\left(\dfrac{h}{9}\right)+
            \mathcal{O}\left(h^{3}\right)
        \end{align*}
        % El término principal de la serie de errores, $K_{2}h^{2}$, se puede eliminar entre las ecuaciones
    \end{solution}
\end{frame}